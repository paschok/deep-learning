\documentclass{scrartcl}
\usepackage[utf8]{inputenc}
\usepackage[english]{babel}
\usepackage[document]{ragged2e}

\setlength{\parskip}{0.6em}
\renewcommand{\baselinestretch}{1.1}

\usepackage{amssymb}
\usepackage{amsmath}
\usepackage{geometry}
\usepackage{float}
 
\begin{document}
\title{Übung zu Deep Learning, SoSe 21}
\subtitle{Tutorial 02: Mathematical Background}
\author{Pavlo Mospan and Anastasia Karsten}
\date{24.04.2021} 
\maketitle

%------------------------------------------------------------
%------------------------------------------------------------
\section{Linear Algebra}

\subsection{Multiple Transpositions}

\textbf{Task :} \textit{Simplify $(A^T)^T$. Justify your solution for instance using the index notation of matrices!}

\textbf{Solution :} 
The transpose of a matrix is an operator which flips a matrix over its diagonal; that is, it switches the row (i) and column (j) indices of the matrix: $A_{ij} = (A^T)_{ji}$

The $(j,i)$-entry of $A^T$ is the $(i,j)$-entry of A, so the $(i,j)$-entry of $(A^T)^T$ is the $(j,i)$-entry of $A^T$, which is the $(i,j)$-entry of A. Thus all entries of $(A^T)^T$ coincide with the corresponding entries of A.

\textbf{So, the transpose of the transpose of A is A, i.e. \((A^T)^T = A\)}

%------------------------------------------------------------
\subsection{Transposing a Matrix Product 2}
\textbf{Task :} \textit{Show that $(AB)^T = B^T A^T$ for a matrices $A \in R^{m \times n}$ and $B \in R^{n \times p}$. What can be said about the dimensions of $(AB)^T$?}

\textbf{Solution :} 

\textit{Option 1}: If $A = 
\begin{vmatrix}
1 & 2\\
3 & 4
\end{vmatrix}$
and $B = 
\begin{vmatrix}
1 & 0 \\
2 & -1
\end{vmatrix}$
, then the product $(AB) =
\begin{vmatrix}
5 & -2 \\
11 & -4
\end{vmatrix}$.

And the transpose of (AB) is: $(AB)^T =
\begin{vmatrix}
5 & 11 \\
-2 & -4
\end{vmatrix}$.

If we take the transpose of A and B separately and multiply A with B, then we have:
 $B^T A^T =
\begin{vmatrix}
1 & 2 \\
0 & -1
\end{vmatrix}
\begin{vmatrix}
1 & 3 \\
2 & 4
\end{vmatrix} 
= 
\begin{vmatrix}
5 & 11 \\
-2 & -4
\end{vmatrix}$.
So, $(AB)^T = B^T A^T$

\textit{Option 2} : We can show, that 
$((AB)_{ij})^T = (\sum_{l=1}^{n} a_{il}b_{lj})^T = \sum_{l=1}^{n} b_{jl}a_{li}$

\textbf{Answer to the second question:} Dimensions of $(AB)^T \in R^{p \times m}$

\subsection{Brackets in Matrix Multiplications} 
\textbf{Task 1: } \textit{How many operations (additions or multiplications of scalars) does a trivial implementation of a matrix multiplication $AB$ with $A \in R^{m \times n}$ and $B \in R^{n \times p}$ need?}

\textbf{Solution:} Multiplications: $n * m * p$. Additions: $(n-1) * (m  * p)$

\textbf{Task 2: } \textit{How many operations does $(AB)C$ with $A \in R^{16 \times 2}$, $B \in R^{2 \times 4}$ and $C \in R^{4 \times 8}$ need? Use the associative property of the matrix multiplication to find the fastest solution.}

\textbf{Solution: }  Associative property of the matrix multiplication : $(AB)C = A(BC)$

\textbf{Let's calculate for} \( \textbf{(AB)C}\).

MULT: $AB = 2 * 16 * 4 = 128$. $(AB)C = 4 * 16 * 8 = 512$. MULT Total: \textbf{640}

ADD: $AB = 16 * 4 = 64$. $(AB)C = 3 * 16 * 8 = 384$. ADD Total: \textbf{448}

\textbf{Let's calculate for} \(\textbf{A(BC)} \).

MULT: $BC = 4 * 2 * 8 = 64$. $A(BC) = 2 * 16 * 8 = 256$. MULT Total: \textbf{320}

ADD: $BC = 3 * 2 * 8 = 48$. $A(BC) = 16 * 8 = 128$. ADD Total: \textbf{176}

\textbf{The fastest solution will take 320 multiplications, 176 additions}

%------------------------------------------------------------
%------------------------------------------------------------
\section{Differential Calculus}
%------------------------------------------------------------
\subsection{Quotient Rule}
\textbf{Task :} \textit{Derive the Quotient rule \[f(x) = \frac{g(x)}{h(x)} \rightarrow \frac{d}{dx}f(x) = \frac{h(x)\frac{d}{dx}g(x) - g(x)\frac{d}{dx}h(x)}{h(x)^2}\] 
using the chain rule and the product rule of differentiation.}

\textbf{Solution :}
\[\frac{d}{dx}\frac{g(x)}{h(x)} = \frac{d}{dx}(g(x) h(x)^{-1}) \]
\[ = \frac{d}{dx}g(x) h(x)^{-1} + g(x) (-1) h(x)^{-2} \frac{d}{dx}h(x)\]
\[= \frac{\frac{d}{dx}g(x)}{h(x)} - \frac{g(x)\frac{d}{dx}h(x)}{h(x)^{2}}\]
\[ =  \frac{\frac{d}{dx}g(x)h(x)}{h(x)^{2}} - \frac{g(x)\frac{d}{dx}h(x)}{h(x)^{2}}\]

\[= \frac{\frac{d}{dx}g(x)h(x) - g(x)\frac{d}{dx}h(x)}{h(x)^2}\]

%------------------------------------------------------------------
\subsection{Derivative of the Sigmoid Function}
\textbf{Task :} \textit{Derive the so-called sigmoid function
\[\sigma(x) = \frac{1}{1+e^{-x}} \]
and express the derivative in terms of \( \sigma(x) \).}

\[\frac{d}{dx}\sigma(x) = \frac{d}{dx}(\frac{1}{1+e^{-x}})\]
\[= \frac{(1+e^{-x})\frac{d}{dx}(1) - (1)\frac{d}{dx}(1+e^{-x})}{(1+e^{-x})^2}\]
\[= \frac{(1+e^{-x})(0) - (1)(-e^{-x})}{(1+e^{-x})^2}\]
\[= \frac{e^{-x}}{(1+e^{-x})^2}\]
\[= \frac{-1+1+e^{-x}}{(1+e^{-x})^2}\]
\[= \frac{1+e^{-x}}{(1+e^{-x})^2} - \frac{1}{(1+e^{-x})^2}\]
\[= \frac{1}{(1+e^{-x})} - \frac{1}{(1+e^{-x})^2}\]
\[= \frac{1}{(1+e^{-x})}(1 - \frac{1}{(1+e^{-x})}) \]
\[= \sigma(x)(1 - \sigma(x))\]

%------------------------------------------------------------------
\subsection{Applying Gradients}
\textbf{Task :} Calculate the gradient of the function \[f(x) = x_1 + x_2 \]

For a given point \textbf{a}, by how much does the value of \textbf{f} change, if we change \textbf{a} by a magnitude of \(\varepsilon (0 < \varepsilon \ll 1) \)  in the direction of the gradient, i.e., calculate

\[ \Delta = f(\textbf{a} + \epsilon) - f(\textbf{a})\]

with 

\[ \epsilon = \varepsilon \frac{\bigtriangledown f(x) \vert_{x=a}}{\Vert \bigtriangledown f(x) \vert_{x=a} \Vert_2}   \]

Compare this to a change of \textbf{a} with magnitude of \( \varepsilon \) in the direction of $x_1$ or $x_2$. How
can the difference be explained qualitatively?

\textbf{Solution :} Let's find out \( \epsilon \) :

\( \bigtriangledown f(x) \vert_{x=a} \) is a vector of first order partial derivatives : \( \bigtriangledown f(x) = \big[\frac{df}{dx_1}; \frac{df}{dx_2} \big] = \big[ 1; 1 \big]\)

\(\Vert \bigtriangledown f(x) \vert_{x=a} \Vert_2 \) is $l^2$ norm of a vector of 1st order partial derivatives \( \Rightarrow  \Vert\bigtriangledown f(x) \Vert_2 = \sqrt{2} \)

So, \( \epsilon = \varepsilon \frac{(1; 1)}{\sqrt{2}} = \varepsilon \frac{1}{\sqrt{2}} \)

Thereof \[ \Delta = f(\textbf{a} + \varepsilon \frac{1}{\sqrt{2}}) - f(\textbf{a})\]

\( \Delta \) is changes by the magnitude of \( \varepsilon \)


\end{document}
